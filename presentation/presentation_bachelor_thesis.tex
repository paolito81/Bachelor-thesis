\documentclass [xcolor=svgnames] {beamer} 
\usepackage[utf8]{inputenc}
\usepackage{xcolor}
\usepackage{booktabs, comment} 
\usepackage{pgfpages}
\usepackage{csquotes}
\usepackage{amsmath}
\usepackage{tikz}
\usetheme{Madrid}

% COLORS 
\definecolor{mqred}{RGB}{166, 25, 46}
\definecolor{mqdeepred}{RGB}{118, 35, 47}
\definecolor{mqgray}{RGB}{55, 58, 54}
\definecolor{mqlightgray}{RGB}{237, 235, 229}
\definecolor{mqmagenta}{RGB}{198, 0, 126}
\usecolortheme[named=mqred]{structure}
\setbeamercolor{title in head/foot}{bg=mqlightgray, fg=mqgray}
\setbeamercolor{author in head/foot}{bg=mqdeepred}
\setbeamercolor{page number in head/foot}{bg=mqdeepred, fg=mqlightgray}

% FOOTNOTE ARRANGEMENTS

\makeatletter
\setbeamertemplate{footline}{
	\leavevmode%
	\hbox{%
		\begin{beamercolorbox}[wd=.5\paperwidth,ht=2.25ex,dp=1ex,center]{author in head/foot}%
			\usebeamerfont{author in head/foot}\insertshortauthor\expandafter\ifblank\expandafter{\beamer@shortinstitute}{}{~~(\insertshortinstitute)}
		\end{beamercolorbox}%
		\begin{beamercolorbox}[wd=.4\paperwidth,ht=2.25ex,dp=1ex,center]{title in head/foot}%
			\usebeamerfont{title in head/foot}\insertshorttitle
		\end{beamercolorbox}%
		\begin{beamercolorbox}[wd=.1\paperwidth,ht=2.25ex,dp=1ex,center]{page number in head/foot}%
			\usebeamerfont{page number in head/foot}\insertframenumber{} / \inserttotalframenumber 
	\end{beamercolorbox}}%
	\vskip0pt%
}
\makeatother
\beamertemplatenavigationsymbolsempty


% TITLE, AUTHORS, INSTITUTE, DATE

\title[Short Title]{Caratterizzazione di un rivelatore gamma 4$\pi$ per lo studio della reazione 14N(p,$\gamma$)15O \\ \textit{Subtitle or Reference, if any}}
\author[P. Pusterla]{Paolo Pusterla}
\institute[UniTo]{Università degli Studi di Torino}
\date{Novembre 2024}

% LOGO
\titlegraphic{\includegraphics[height=2.5cm]{img/logo.png}} % Change the logo path as needed

\begin{document}
	
	\begin{frame}
		\titlepage
	\end{frame}
	
	\begin{frame}{Outline}
		\tableofcontents
	\end{frame}
	
	% Section and Frame examples
	\section{Introduzione}
	\begin{frame}{L'esperimento}
		\begin{itemize}
			\item L'esperimento LUNA (Laboratory for Underground Nuclear Astrophysics) ricrea i processi nucleari che sono avvenuti durante la nucleosintesi primordiale e che avvengono tutt'ora nelle stelle.
			\item Essendo processi molto rari, un laboratorio sulla superficie terrestre non è adatto per le misure sperimentali di questi, poiché i raggi cosmici maschererebbero il segnale debole atteso.
			\item Per questo motivo i Laboratori Nazionali del Gran Sasso sono i luoghi adatti per questi esperimenti: le sale sperimentali in cui si effettuano sono protette e schermate dai 1400 m di roccia del monte Aquila. 
		\end{itemize}
	\end{frame}
	
	\begin{frame}{La reazione}
		\begin{itemize}
			\item Viene studiata la reazione $^{14}$N(p,$\gamma$)$^{15}$O del ciclo CNO, determinante per la produzione di neutrini solari.
			\item La misura della sezione d'urto di questa reazione ha portato alla riduzione di un fattore 2 del flusso di neutrini solari prodotti dal ciclo CNO.
			\item Ha inoltre permesso di aggiornare la stima dell'età della Via Lattea.
		\end{itemize}
	\end{frame}

	\begin{frame}{Il ciclo CNO}
		\begin{figure}[H]
			\includegraphics[width=0.5\textwidth]{img/CNO_Cycle.pdf}
			\caption{Ciclo Carbonio-Azoto-Ossigeno}
		\end{figure}
	\end{frame}

	\begin{frame}{L'acceleratore LUNA2 400 kV}
		\begin{itemize}
			\item Il fascio ionico utilizzato nell'esperimento è fornito dall'acceleratore elettrostatico LUNA2 a 400 kV.
			\item Installato nel 2001, ha soppiantato il precedente acceleratore di 50 kV (utilizzato sino al 2003) e fornisce fasci di ioni molto più intensi e temporalmente stabili.
		\end{itemize}
	\end{frame}
	
	\section{Obiettivi della tesi}
	\begin{frame}{Obiettivi}
		\begin{itemize}
			\item L'obiettivo della tesi è quello di caratterizzare in efficienza uno scintillatore $4\pi$ utilizzato per la rivelazione di raggi $\gamma$ nella riproduzione della reazione $^{14}$N(p,$\gamma$)$^{15}$O. 
			\item Include relevant equations, graphs, or figures.
		\end{itemize}
	\end{frame}
	
	\section{Il rivelatore}
	\begin{frame}{Il rivelatore $4\pi$}
		\begin{itemize}
			\item Lo scintillatore utilizzato è un rivelatore in germanato di bismuto (Bi$_{4}$Ge$_{3}$O$_{12}$, detto BGO).
			\item Il cristallo, di forma cilindrica, è otticamente separato in 6 spicchi diversi.
			\item I fotoni di scintillazione di ciascuno dei 6 segmenti sono rivelati da due fotomoltiplicatori.
		\end{itemize}
	\end{frame}

\begin{frame}{Il rivelatore $4\pi$}
	\begin{figure}[h]
		\includegraphics[width=0.8\textwidth]{img/BGO.png}
		\caption{Rappresentazione schematica del rivelatore BGO. In alto una sezione sagittale, in basso una sezione trasversale.}
	\end{figure}
\end{frame}

\section{Richiami teorici}
\begin{frame}{Sezione d'urto}
	\begin{itemize}
		\item La \emph{sezione d'urto} di una reazione nucleare è una grandezza utilizzata per descrivere la probabilità che la reazione avvenga.
		\item La si può calcolare sperimentalmente come:
		
%		\begin{equation}
%			\sigma = \dfrac{\mathrm{numero di interazione per unità di tempo}}{(\mathrm{numero di particelle incidenti per unità di tempo})\cdot(\mathrm{numero di nuclei bersaglio all'interno del fascio incidente})}
%		\end{equation}
		
	\end{itemize}
\end{frame}

\begin{frame}{Efficienza}
	\begin{itemize}
		\item L'efficienza di uno scintillatore è il rapporto tra il numero di conteggi prodotti da esso e il numero di conteggi prodotti dalla sorgente:
		
		\begin{equation}
			\varepsilon = \dfrac{N_{\gamma}}{N_{int}}
		\end{equation}
		
		\item Si tratta pertanto di quanti fotoni lo strumento "vede" rispetto al totale
		\item Invertendo l'equazione possiamo ricavare $N_{int}$, per poi trovare la sezione d'urto
	\end{itemize}
\end{frame}

\begin{frame}{Efficienza}
	\begin{itemize}
		\item Nel caso dei nostri istogrammi, composti da un picco gaussiano centrato su un'energia caratteristica e un fondo di \emph{bremsstrahlung}, l'efficienza si può calcolare nel modo seguente:
		
		\begin{equation}
			\varepsilon = \dfrac{N_{cont.}}{A(t*) \Delta t}
		\end{equation}
	
		dove $N_{cont.}$ è il numero di conteggi del picco gaussiano sottraendone il fondo, $A(t*)$ è l'attività calcolata al momento della misura, $\Delta t$ è il tempo "vivo" dello strumento.
	\end{itemize}
\end{frame}

\section{L'elettronica}

\begin{frame}{Tempo vivo/morto}
	\begin{itemize}
		\item Ogni strumento è elettronicamente vincolato a processare il segnale in ingresso
		\item Questo può richiedere fino a ns
		\item Un fotone in arrivo durante questo intervallo di tempo non può essere quindi rilevato
		\item Alla fine della misura verrano osservati meno fotoni di quelli effettivamente giunti allo strumento, perché quest'ultimo è attivo solo per una parte di tempo rispetto al totale della misura.
		\item L'intervallo in cui lo strumento è attivo e pronto a ricevere nuovi segnali è il \emph{tempo vivo}.
	\end{itemize}
\end{frame}

\section{Caratterizzaione}
\begin{frame}{Caratterizzazione in energia}
	\begin{itemize}
		\item La caratterizzazione in energia è effettuata utilizzando due sorgenti radioattive: $^{60}$Co e $^{137}$Cs.
	\end{itemize}
\end{frame}

\begin{frame}{$^{60}$Co}
	\begin{itemize}
		\item Il $^{60}$Co decade tramite decadimento $\beta^{-}$ (99.75\%) in $^{60}$Ni ed emette due raggi gamma di energie 1.17 MeV e 1.33 MeV.
		\item Ha il vantaggio di emettere raggi gamma ad alta intensità con un'emivita relativamente lunga di 5.27 anni. 
		\item Trova applicazione nella radioterapia del cancro.
	\end{itemize}
\end{frame}

\begin{frame}{$^{137}$Cs}
	\begin{itemize}
		\item Il $^{137}$Cs decade sempre tramite decadimento $\beta^{-}$.
		\item Il 94.6\% dei decadimenti hanno come prodotto uno stato metastabile del $^{137}$Ba. 
		\item Questo stato eccitato emette l'85\% delle volte raggi gamma di 661.7 keV decadendo nello stato fondamentale del $^{137}$Ba (tutti i raggi gamma provenienti dal $^{137}$Cs sono prodotti così).
		\item Trova applicazione nella calibrazione di strumenti.
	\end{itemize}
\end{frame}

\begin{frame}
	\begin{itemize}
		\item I modi di decadimento degli elementi utilizzati sono
		\item Schema?? Energie che figurano nel nostro caso
	\end{itemize}
\end{frame}

\begin{frame}{Struttura dei dati}
	\begin{itemize}
		\item I dati ricavati sono contenuti in file .root
		\item Ogni file .root contiene 8 istogrammi, con indici da 0 a 7, di conteggi
		\item L'istogramma 0 contiene il pulser, utilizzato per calcolare il tempo vivo dello scintillatore
		\item Gli istogrammi da 1 a 6 sono i singoli spicchi del BGO
		\item L'istogramma 7 è
	\end{itemize}
\end{frame}

\begin{frame}
	METTERE FOTO DI UN ISTOGRAMMA
\end{frame}

\begin{frame}{Calcolo della calibrazione}
	\begin{itemize}
		\item La calibrazione viene effettuata sul file \texttt{run1775\_coinc.root}, con entrambe le sorgenti.
		\item \emph{Calibrare} uno scintillatore significa trovare il fattore di conversione da \textbf{canali} a \textbf{energia}.
		\item Per ogni spicchio del BGO si esegue un fit per trovare il valore dei picchi caratteristici e del picco somma in canali.
	\end{itemize}
\end{frame}

\begin{frame}{Esempio di istogramma}
	METTERE ISTOGRAMMA DEL FILE 1775
\end{frame}

\begin{frame}{Calcolo della calibrazione}
	\begin{itemize}
			\item I valori dei picchi vengono graficati contro quelli in energia caratteristici, noti.
		\item Si effettua dunque un fit lineare per i punti di ciascun istogramma.
		\item I coefficienti angolari di questi fit sono i fattori di conversione cercati.
	\end{itemize}
\end{frame}

\begin{frame}{Grafici}
	METTERE GRAFICI DEI PLOT DELLA CALIBRAZIONE
\end{frame}

\begin{frame}{Calcolo dell'efficienza}
	\begin{itemize}
		\item Per il calcolo dell'efficienza si possono applicare due metodi diversi
		\item Il primo metodo è geometrico, il secondo riguarda i parametri del fit sui picchi caratteristici.
		\item Il metodo geometrico trova applicazione nel cesio, ma poiché il cobalto presenta picchi sovrapposti, non è applicabile a quest'ultimo.
	\end{itemize}
\end{frame}
	
	\section{Conclusion}
	\begin{frame}{Conclusion}
		\begin{itemize}
			\item Summarize the key takeaways from your presentation.
			\item Mention any future work or open questions.
		\end{itemize}
	\end{frame}
	
	\begin{frame}
		\centering
		\textbf{Thank you!}\\
		Questions?
	\end{frame}
	
\end{document}

